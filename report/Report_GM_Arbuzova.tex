\documentclass[12pt,a4paper,oneside,fleqn,leqno]{article}
\usepackage[utf8]{inputenc}
\usepackage[T1]{fontenc}
\usepackage{geometry}
\usepackage{graphicx}
\usepackage{amssymb}
\usepackage{amsmath}
\usepackage{wrapfig}
\usepackage{float}
%\usepackage{caption}
%\usepackage{subcaption}
\usepackage{listings}
\usepackage[center]{caption}
\geometry{a4paper}
\usepackage[russian]{babel}

\setlength{\topmargin}{-1.0in}
\setlength{\textheight}{10.5in}
\setlength{\oddsidemargin}{-.125in}
\setlength{\textwidth}{6.5in}

\begin{document}
	\begin{titlepage}
		\begin{center}
			\large Московский государственный университет им.\,М.\,В.\,Ломоносова\\
			Факультет вычислительной математики и кибернетики \\
			Кафедра математических методов прогнозирования \\[4.5cm] 
			\Huge Задание \No\,1.\\Вероятностные модели посещаемости курса\\[5.5cm]
		\end{center}
		\normalsize
		\begin{flushright}
			\emph{Автор:} Арбузова Дарья\\
			\emph{Группа:} 417\\
		\end{flushright}
		\vfill
		\begin{center}
			Москва, 2014
		\end{center}
	\end{titlepage}
	\section{Цели задания}
		В рамках данного задания мы познакомимся с байесовскими сетями…
	\section{Описание моделей}
		Рассмотрим модель посещаемости студентами одного курса лекции. Пусть аудитория данного курса состоит из студентов профильной кафедры, а также студентов других кафедр. Обозначим через $a$ количество студентов, распределившихся на профильную кафедру, а через $b$ --- количество студентов других кафедр на курсе. Пусть студенты профильной кафедры посещают курс с некоторой вероятностью $p_1$, а студенты остальных кафедр --- с вероятностью $p_2$. Обозначим через $c$ количество студентов на данной лекции. Тогда случайная величина $c|a,b$ есть сумма двух случайных величин, распределенных по биномиальному закону $B(a,p_1)$ и $B(b,p_2)$ соответственно. Пусть далее на лекции по курсу ведется запись студентов. При этом каждый студент записывается сам, а также, быть может, записывает своего товарища, которого на лекции на самом деле нет. Пусть студент записывает своего товарища с некоторой вероятностью $p_3$. Обозначим через $d$ общее количество записавшихся на данной лекции. Тогда случайная величина $d|c$ представляет собой сумму c и случайной величины, распределенной по биномиальному закону $B(c,p_3)$. Для завершения задания вероятностной модели осталось определить априорные вероятности для $a$ и для $b$. Пусть обе эти величины распределены равномерно в своих интервалах $[a_{min}; a_{max}]$ и $[b_{min}; b_{max}]$ (дискретное равномерное распределение). Таким образом, мы определили следующую вероятностную модель: \par
		Рассмотрим несколько упрощенную версию модели 1. Известно, что биномиальное распределение $B(n,p)$ при большом количестве испытаний и маленькой вероятности успеха может быть с высокой точностью приближено пуассоновским распределением $Poiss(\lambda)$ с $\lambda = np$. Известно также, что сумма двух пуассоновских распределений с параметрами $\lambda_1$ и $\lambda_2$ есть пуассоновское распределение с параметром $\lambda_1+\lambda_2$ (для биномиальных распределений аналогичное неверно). Таким образом, мы можем сформулировать вероятностную модель, которая является приближенной версией модели 1:
	
	\section{Модель 1}
		\subsection{Вывод формул для расчёта распределений}
			Проведём вывод необходимых распределений для рассматриваемой модели, пользуясь фактами из теории вероятностей:
			\begin{description}
				\item[$a$]
				$\sim R[a_{min}; a_{max}]$\\
				$$p(a) = \frac{1}{a_{max} - a_{min} + 1}$$
				$$\mathbb{E}[a] = \frac{a_{min} + a_{max}}{2}$$
				$$\mathbb{D}[a] = \frac{(a_{max} - a_{min} + 1)^2 - 1}{12}$$
				\noindent\makebox[\linewidth]{\rule{\textwidth}{0.4pt}}
				\item[$b$]
				$\sim R[b_{min}; b_{max}]$ аналогично:\\
				$$p(b) = \frac{1}{b_{max} - b_{min} + 1}$$
				$$\mathbb{E}[b] = \frac{b_{min} + b_{max}}{2}$$
				$$\mathbb{D}[b] = \frac{(b_{max} - b_{min} + 1)^2 - 1}{12}$$
				\noindent\makebox[\linewidth]{\rule{\textwidth}{0.4pt}}
				\item[$p(b|a)$]
				$$ p(b|a) = \frac{p(a, b)}{p(a)} = \frac{\sum\limits_{c = 0}^{a + b} \sum\limits_{d = c}^{2c} p(a, b, c, d)}{p(a)} = p(b) \cdot \sum\limits_{c = 0}^{a + b} p(c|a, b) \cdot \sum\limits_{d = c}^{2c} p(d|c) = p(b)$$
				Величины $a$ и $b$ независимы в рамках данной модели.\\
				\noindent\makebox[\linewidth]{\rule{\textwidth}{0.4pt}}
				\item[$c|a,b$]
				$\sim B(a, p_1) + B(b, p_2)$\\
				Пусть $c = x + y,$ где $x \sim B(a, p_1), y \sim B(b, p_2),$ тогда
				$$p(c|a,b) = \sum_{k = 0}^c p(x = k; a, p_1) \cdot p(y = c - k; b, p_2) = $$ $$
				=\sum_{k = 0}^c C_a^k p_1^k (1 - p_1)^{a - k} \cdot C_b^{c - k} p_2^{c - k} (1 - p_2)^{b + k - c}$$
				\noindent\makebox[\linewidth]{\rule{\textwidth}{0.4pt}}
				\item[$d|c$]
				$\sim c + B(c, p_3)$
				$$ p(d|c) = C_{c}^{d - c}p_3^{d - c}(1 - p_3)^{2c - d}$$
				\noindent\makebox[\linewidth]{\rule{\textwidth}{0.4pt}}
				\item[$c|a$]
				$$p(c|a) = \frac{p(a, c)}{p(a)} = \frac{\sum\limits_{b = b_{min}}^{b_{max}} \sum\limits_{d = 0}^{2(a + b)}p(a, b, c, d) }{p(a)} = $$ $$ = p(b) \cdot \sum\limits_{b = b_{min}}^{b_{max}} p(c|a, b) \cdot \sum\limits_{d = 0}^{2(a + b)}p(d|c) = p(b) \cdot \sum\limits_{b = b_{min}}^{b_{max}} p(c|a, b)$$
				\noindent\makebox[\linewidth]{\rule{\textwidth}{0.4pt}}
				\item[$c|b$] аналогично:
				$$p(c|b) = p(a) \cdot \sum\limits_{a = a_{min}}^{a_{max}} p(c|a, b)$$
				\noindent\makebox[\linewidth]{\rule{\textwidth}{0.4pt}}
				\item[$c$]
				$$p(c) = \sum\limits_{a = a_{min}}^{a_{max}} \sum\limits_{b = b_{min}}^{b_{max}} \sum\limits_{d = 0}^{2(a + b)} p(a, b, c, d) =$$ 
				$$ = p(a) \cdot p(b) \cdot \sum\limits_{a = a_{min}}^{a_{max}} \sum\limits_{b = b_{min}}^{b_{max}} p(c|a,b) \cdot \sum\limits_{d = 0}^{2(a + b)} p(d|c) = p(a)\cdot p(b) \cdot  \sum\limits_{a = a_{min}}^{a_{max}} \sum\limits_{b = b_{min}}^{b_{max}} p(c|a,b)$$
				\noindent\makebox[\linewidth]{\rule{\textwidth}{0.4pt}}
				\item[$d$]
				$$ p(d) = \sum\limits_{c = 0}^{a_{max} + b_{max}}p(d|c)p(c)$$
				\noindent\makebox[\linewidth]{\rule{\textwidth}{0.4pt}}
			\end{description}
		\subsection{Априорные распределения}
			Требуется рассчитать математические ожидания и дисперсии априорных распределений $a, b, c$ и $d$.\par
			Пусть для некоторый случайной величины $x$ известно её распределение, тогда
			$$ \mathbb{E}[x] = \sum_{x = x_{min}}^{x_{max}}xp(x)$$
			$$ \mathbb{D}[x] = \sum_{x = x_{min}}^{x_{max}}x^2p(x) - (\mathbb{E}[x])^2$$\par
			В пункте 3.1 было показано, как получить априорные распределения, имея $p(c|a,b)$ и $p(d|c).$\par
			Результаты приведены в таблице \ref{tab:aprior1}:
			\begin{table}[H]
			\centering
			\begin{tabular}{|c|c|c|}
				\hline
				Величина & $\mathbb{E}$ & $\mathbb{D}$\\
				\hline
				a & 22.5 & 21.25\\
				\hline
				b & 300 & 850\\
				\hline
				c & 26.25 & 27.3125\\
				\hline
				d & 39.375 & 68.0156\\
				\hline
				\end{tabular}
			\caption{Априорные распределения}
			\label{tab:aprior1}
		\end{table}
		\subsection{Прогноз величины $b$}
			Требуется пронаблюдать, как происходит уточнение прогноза для величины $b$ с приходом новой информации.\par
			Рассмотрим распределения $b, b|a, b|a,d$. Как было показано выше в пункте 3.1, $b$ и $b|a$ распределены одинаково, поэтому остаётся сравнить только $b$ с $b|a, d$.\par
			На рисунке ? показано распределение соответствующих величин, а также указаны их матожидания и дисперсии.
		\subsection{Влияние параметров $p_1$ и $p_2$}
		\subsection{Временные замеры}
	\section{Модель 2}
		\subsection{Априорные распределения}
			Аналогично пункту для модели 1.\par
			Результаты приведены в таблице \ref{tab:aprior2}:
			\begin{table}[H]
			\centering
			\begin{tabular}{|c|c|c|}
				\hline
				Величина & $\mathbb{E}$ & $\mathbb{D}$\\
				\hline
				a & 22.5 & 21.25\\
				\hline
				b & 300 & 850\\
				\hline
				c & 26.25 & 33.6875\\
				\hline
				d & 39.375 & 82.3594\\
				\hline
				\end{tabular}
			\caption{Априорные распределения}
			\label{tab:aprior2}
		\end{table}
		\subsection{Прогноз величины $b$}
		\subsection{Влияние параметров $p_1$ и $p_2$}
		\subsection{Временные замеры}
	\section{Сравнение моделей 1 и 2}
		Модель 2 получена из первой путём предельного перехода… Значит, первая описывает всё лучше и правильней при малых значениях.
	\section{Выводы}
		Печаль-беда
	\section{Список литературы}
		\begin{itemize}
			\item
			http://www.machinelearning.ru/wiki/index.php?title=
			\item
			Лекции и семинары по графическим моделям
		\end{itemize}
\end{document}